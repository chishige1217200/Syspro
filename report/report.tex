% ファイル先頭から\begin{document}までの内容(プレアンブル)については,
% 基本的に { } の中を書き換えるだけでよい.
\documentclass[autodetect-engine,dvi=dvipdfmx,ja=standard,
               a4j,11pt]{bxjsarticle}

%%======== プレアンブル ============================================%%
% 用紙設定:指示があれば,適切な余白に設定しなおす
\RequirePackage{geometry}
\geometry{reset,a4paper}
\geometry{hmargin=25truemm,top=25truemm,bottom=25truemm,footskip=10truemm}
%\geometry{showframe} % 本文の"枠"を確認したければ,コメントアウト

% 設定:図の挿入
% http://www.edu.cs.okayama-u.ac.jp/info/tool_guide/tex.html#graphicx
\usepackage{graphicx}

% 設定:ソースコードの挿入
% http://www.edu.cs.okayama-u.ac.jp/info/tool_guide/tex.html#fancyvrb
\usepackage{fancyvrb}
\renewcommand{\theFancyVerbLine}{\texttt{\footnotesize{\arabic{FancyVerbLine}:}}}

%%======== レポートタイトル等 ======================================%%
% ToDo: 提出要領に従って,適切なタイトル・サブタイトルを設定する
\title{システムプログラミング1 \\
       期末レポート}

% ToDo: 自分自身の氏名と学生番号に書き換える
\author{氏名: 重近 大智 (SHIGECHIKA, Daichi) \\
        学生番号: 09501527}

% ToDo: レポート課題等の指示に従って適切に書き換える
\date{出題日: 2020年10月05日 \\
      提出日: 2020年11月xx日 \\
      締切日: 2020年11月16日 \\}  % 注:最後の\\は不要に見えるが必要.


%%======== 本文 ====================================================%%
\begin{document}
\maketitle
% 目次つきの表紙ページにする場合はコメントを外す
%{\footnotesize \tableofcontents \newpage}

%% 本文は以下に書く.課題に応じて適切な章立てを構成すること.
%% 章=\section,節=\subsection,項=\subsubsection である.

%--------------------------------------------------------------------%
\section{概要} \label{sec:abstract}
本レポートでは,MIPS言語を用いて,提示された5つの課題に取り組み,その解答を報告する.実行結果はxspimによる結果である.

本レポートで報告するシステムプログラミング1の課題は次の5つである.
\begin{enumerate}
\item A.8節 「入力と出力」に示されている方法と, A.9節 最後「システムコール」に示されている方法のそれぞれで \verb|"Hello World"| を表示せよ.両者の方式を比較し考察せよ.\cite{book:assembly}
\item アセンブリ言語中で使用する \verb|.data|, \verb|.text| および \verb|.align| とは何か解説せよ. \ref{sec:p1-2}節のコード中の 9行目の \verb|.data| がない場合,どうなるかについて考察せよ.
\item A.6節 「手続き呼出し規約」に従って,再帰関数 \verb|fact| を実装せよ. (以降の課題においては,この規約に全て従うこと)\cite{book:assembly}
\item 素数を最初から100番目まで求めて表示するMIPSのアセンブリ言語プログラムを作成してテストせよ. その際,素数を求めるためにに表\ref{tab:routine}に示す2つのルーチンを作成すること.
\item 素数を最初から100番目まで求めて表示するMIPSのアセンブリ言語プログラムを作成してテストせよ. ただし,配列に実行結果を保存するように \verb|main| 部分を改造し, ユーザの入力によって任意の番目の配列要素を表示可能にせよ.
\end{enumerate}

\begin{table}[b]
\centering
	\caption{課題4で実装する2つのルーチン}
	\label{tab:routine}
    	\begin{tabular}{|l|l|}
	\hline
関数名&概要\\
	\hline
\verb|test_prime(n)|&\verb|n|が素数なら$1$,そうでなければ$0$を返す\\
	\hline
\verb|main()|&整数を順々に素数判定し,100個プリント\\
	\hline
	\end{tabular}
\end{table}

%--------------------------------------------------------------------%
\section{プログラムの説明}\label{sec:capp}
使用したMIPSアセンブリ言語のソースコードは,\ref{sec:makep}章に示す.

\subsection{課題1-1}
まず,A.8節 「入力と出力」に示されている方法で実装したコードについて説明する.このプログラムは起動すると,\verb|msg|ラベルに対応するメモリ上のアドレスを\verb|la|命令により,\verb|$a0|レジスタにロードする.次に\verb|move|命令を用いて,\verb|main|を呼び出したコンソールに戻るためのアドレスを\verb|$s0|レジスタに退避する.続いて,\verb|loop|ラベルから始まるループ部に突入する.ここでは\verb|lb|命令を用いて,\verb|$a0|レジスタが指すメモリ上のアドレスに記録されているバイトを\verb|$a1|レジスタにロードする.ロードしたときの値が$0$であった場合,\verb|beqz|命令での分岐条件を満たし,\verb|end|ラベルに対応するメモリ上のアドレスの命令にジャンプする.条件を満たさない場合は,\verb|jal|命令で\verb|print|ラベルにジャンプするとともに,\verb|$ra|レジスタに\verb|jal print|の次の命令が存在するメモリ上のアドレスを代入する.\verb|print|ラベルから始まる表示の処理では,まず\verb|lw|命令を用いて,メモリ上のアドレス\verb|0xffff0008|に存在するワードを\verb|$t0|レジスタにロードする.このとき,\verb|$t0|レジスタに入っている値は$1$かそれ以外であり,これが$1$であればプリンタは動作可能な状態である.\verb|li|命令で,即値$1$を\verb|$t1|レジスタにロードして,\verb|and|命令で\verb|$t0|レジスタと\verb|$t1|レジスタの論理積をとって\verb|$t0|レジスタに代入する.次の\verb|beqz|命令で\verb|$t0|レジスタの値が$0$である場合,プリンタの準備はできていないものとし,もう一度\verb|print|ラベルに対応するメモリ上のアドレスの命令に戻って準備を試みる.準備ができてい場合は,\verb|sw|命令でメモリ上のアドレス\verb|0xffff000c|に表示したい$1$文字の\verb|ASCII|コードを送る.送った瞬間にSPIMのコンソールに文字が表示される.そして,\verb|j $ra|により呼び出し元である\verb|main|中の\verb|jal print|の次の命令にジャンプする.ここまでで1文字表示できたので,次の文字情報を得るために\verb|addi|命令で\verb|$a0|レジスタの




\subsection{課題1-2}


\subsection{課題1-3}


\subsection{課題1-4}


\subsection{課題1-5}



%--------------------------------------------------------------------%
\section{プログラムの使用法と実行結果}\label{sec:howresult}

プログラムは,CentOS 7.6.1810 (Core) のxspimで動作を確認しているが,
一般的な UNIX で動作することを意図している.
まず,ターミナルに\verb|xspim -mapped_io&|と打ち込んで,xspimを実行する.
実行後にloadの機能を使い,拡張子が\verb|.s|のアセンブリファイルを読み込む.runの機能で読み込んだプログラムを走らせる.


\subsection{課題1-1}


\subsection{課題1-2}


\subsection{課題1-3}


\subsection{課題1-4}


\subsection{課題1-5}


%--------------------------------------------------------------------%
\section{考察} \label{sec:review}

\subsection{課題1-1}
A.8節 「入力と出力」に示されている方法に示されている方法で実装する場合,空行を除くコードは25行である.
それに対して,A.9節 最後「システムコール」に示されている方法で実装する場合,空行を除くコードはたった12行となった.
これは前者が1文字ずつしか表示できないため,1文字ずつ読み込む処理とアドレスを$1$加える処理(12~16行目)がループになっていることと,プリンタを利用する際にプリンタが使用可能な状態になっていることを確認するために\verb|0xffff0008|番地から値を読み,$1$になっているか確認する処理(18~21行目)があることに起因する.また\verb|jal|命令を用いているため,\verb|main|を呼び出したときの\verb|$ra|レジスタの値が破壊されるため,そのコピーをするために\verb|move|命令を使っていることも行数が増える原因の一端となっている.後者は,文字列を表示できるため,アドレスに1を加えてずらす処理が必要なく,表示するにあたって特定番地の値の確認を取る必要もない.必要なのは,\verb|syscall|命令の動作を規定する引数を\verb|$v0|レジスタに事前に代入するだけである.そのため,非常に簡潔なコードとなっている.

特に 「入力と出力」に示されている方法で実装する場合には,\verb|0xffff0008|番地の確認を怠ると,プリンタの故障につながることがあったり,計算機によって異なるアドレスをとる可能性のある\verb|0xffff0008|番地をプログラムのコード内に書いてしまうことになったりするなどの問題もある.また,他のプログラムと同時に走行した場合の優先順序をつけられないために競合が発生することがある.システムコールを用いていれば,表示するタイミングや他のプログラムとの優先順序もオペレーティングシステムが決めてくれるため,プログラムを書く際に深く意識する必要はない.

\subsection{課題1-2}


%--------------------------------------------------------------------%
\section{感想} \label{sec:review}


%--------------------------------------------------------------------%
\section{作成したプログラムのソースコード} \label{sec:makep}

使用したプログラムを以下に添付する.
%なお,\ref{sec:abstract}章に示した課題については,
%\ref{xxxx}章で示したようにすべて正常に動作したことを付記しておく.

% Verbatim environment
% プリアンブルで \usepackage{fancyvrb} が必要.
%   - numbers           行番号を表示.left なら左に表示.
%   - xleftmargin       枠の左の余白.行番号表示用に余白を与えたい.
%   - numbersep         行番号と枠の間隙 (gap).デフォルトは 12 pt.
%   - fontsize          フォントサイズ指定
%   - baselinestretch   行間の大きさを比率で指定.デフォルトは 1.0.


\subsection{課題1-1で用いたコード} \label{sec:p1-1}
下記は,「入力と出力」に示されている方法で実装した例.
\begin{Verbatim}[numbers=left, xleftmargin=10mm, numbersep=6pt,
                    fontsize=\small, baselinestretch=0.8]
    .data                   # データセグメント
    .align 2                # 2のn乗の境界上になるまで隙間を空けてくれる
msg:
    .asciiz "Hello World"   # 出力文字情報

    .text                   # テキストセグメント
    .align 2                # バイトを揃える
main:
    la      $a0, msg        # a0にmsgのアドレスを格納する
    move    $s0, $ra        # $s0にmainを呼び出した元のアドレスを格納
loop:
    lb      $a1, 0($a0)     # a0の指し先の値を$a1にロードする
    beqz    $a1, end        # a1が0のとき,endへ
    jal     print           # printのアドレスにジャンプ(次の命令のアドレスを$raに)
    addi    $a0, $a0, 1     # $a0に1を加える
    j       loop            # mainのアドレスにジャンプ
print:
    lw      $t0, 0xffff0008 # 0xffff0008番地にあるワードを$t0にロードする
    li      $t1, 1          # $t1に1を代入する
    and     $t0, $t0, $t1   # $t0と$t1の論理積をとって$t0に代入する($t0が1か確認する)
    beqz    $t0, print      # $t0が0のとき,もう一度準備を試みる
    sw      $a1, 0xffff000c # 0xffff000c番地に$a1にあるワードを送る
    j       $ra             # 呼び出し元に戻る
end:
    move    $ra, $s0        # mainを呼び出した元のアドレスを$raに復元
    j       $ra             # コンソールに戻る
\end{Verbatim}

下記は,「システムコール」に示されている方法で実装した例.
\begin{Verbatim}[numbers=left, xleftmargin=10mm, numbersep=6pt,
                    fontsize=\small, baselinestretch=0.8]
    .data                   # データセグメント
    .align 2                # バイトを揃える
msg:
    .asciiz "Hello World"   # 出力文字情報

    .text                   # テキストセグメント
    .align 2                # バイトを揃える
main:
    la      $a0, msg        # msg のアドレスを $a0 に格納
    li      $v0, 4          # print_stringを指定
    syscall                 # システムコールの実行
    j       $ra             # コンソールに戻る
\end{Verbatim}

\subsection{課題1-2で用いたコード} \label{sec:p1-2}
\begin{Verbatim}[numbers=left, xleftmargin=10mm, numbersep=6pt,
                    fontsize=\small, baselinestretch=0.8]
    .text
    .align 2

_print_data:
    la      $a0, ghost
    lw      $a0, 0($a0)
    li      $v0, 1

    .data
    .align 2
ghost:
    .word 12

    .text
    syscall
    j       $ra

main:
    subu    $sp, $sp, 24
    sw      $ra, 16($sp)
    jal     _print_data
    lw      $ra, 16($sp)
    addu    $sp, $sp, 24
    j       $ra
\end{Verbatim}


\subsection{課題1-3で用いたコード} \label{sec:p1-3}
\begin{Verbatim}[numbers=left, xleftmargin=10mm, numbersep=6pt,
                    fontsize=\small, baselinestretch=0.8]
    .text
    .align 2
main:
    move    $s0, $ra        # mainを呼んだ戻り先のアドレスを$s0に保存しておく

    la      $a0, msg        # msgのアドレスを$a0にロード
    jal     print_str       # print_strのアドレスにジャンプ(次の命令のアドレスを$raに)

    li      $a0, 10         # $a0に10を代入(任意のint値)

    jal     fact            # factに飛ぶ
    move    $a0, $v0        # $a0に$v0の値をコピー
    jal     print_int       # print_intのアドレスにジャンプ(次の命令のアドレスを$raに)

    la      $a0, msg2       # msg2のアドレスを$a0にロード
    jal     print_str       # print_strのアドレスにジャンプ(次の命令のアドレスを$raに)

    move    $ra, $s0        # mainを呼んだ戻り先のアドレスを$raに代入
    j       $ra             # コンソールに戻る

fact:
    subu    $sp, $sp, 32    # 32バイト確保
    sw      $ra, 20($sp)    # $sp + 20のアドレスにもともとの呼出しアドレスを保存
    sw      $fp, 16($sp)    # $sp + 16のアドレスに$fpを保存
    sw      $a0, 24($sp)    # $sp + 24のアドレスに$a0を保存
    addu    $fp, $sp, 28    # 新しく$fpを設定

    li      $t0,   1        # $t0に1を代入(分岐命令用)

    bgeu    $a0, $t0, else  # $a0が$t0より小さいとき,elseへ
    li      $v0, 1          # $v0に1を代入
    lw      $ra, 20($sp)    # 保存しておいた$spを復元
    lw      $fp, 16($sp)    # 保存しておいた$fpを復元
    addu    $sp, $sp, 32    # スタックを開放
    j       $ra             # 呼び出し元に戻る
else:
    addi    $a0, -1         # $a0から1を引く
    jal     fact            # 再帰呼び出しをする

    lw      $ra, 20($sp)    # 保存しておいた$spを復元
    lw      $fp, 16($sp)    # 保存しておいた$fpを復元
    lw      $a0, 24($sp)    # 対応する$a0の値を復元
    addu    $sp, $sp, 32    # スタックを開放

    mulo    $v0, $a0, $v0   # $v0と$a0をかけて$v0に代入
    j       $ra             # 呼び出し元に戻る

print_int:
    li      $v0, 1          # $v0に1を代入
    syscall                 # システムコールの実行
    j       $ra             # 呼び出し元に戻る
print_str:
    li      $v0, 4          # $v0に4を代入
    syscall                 # システムコールの実行
    j       $ra             # 呼び出し元に戻る

    .data
    .align 2
msg:
    .asciiz "The factorial of 10 is "

    .align 2
msg2:
    .asciiz "\n"
\end{Verbatim}

\subsection{課題1-4で用いたコード} \label{sec:p1-4}
\begin{Verbatim}[numbers=left, xleftmargin=10mm, numbersep=6pt,
                    fontsize=\small, baselinestretch=0.8]
    .text
    .align 2
main:
    move    $s2, $ra        # $s2にmainを呼び出した元のアドレスを格納
    li      $s0, 2          # $s0に2を格納(検索する数)
    li      $s1, 0          # $s1に0を格納(素数の個数カウント)
loop:
    move    $a0, $s0        # $s0の値を$a0にコピー
    jal     test_prime      # test_primeのアドレスにジャンプ(次の命令のアドレスを$raに)
    beqz    $v0, r1         # $v0が0ならば,r1に分岐
    addi    $s1, 1          # $s1に1を加える
    move    $a0, $s0        # $v0の値を$a0にコピー
    jal     print_int       # print_intのアドレスにジャンプ(次の命令のアドレスを$raに)
    la      $a0, space      # spaceのアドレスを$a0にロード
    jal     print_str       # print_strのアドレスにジャンプ(次の命令のアドレスを$raに)
r1:
    bge     $s1, 100, end   # $s1が100以上のとき,endにジャンプ
    addi    $s0, 1          # $s0に1を加える
    j       loop            # loopのアドレスにジャンプ
end:
    la      $a0, line       # lineのアドレスを$a0にロード
    jal     print_str       # print_strのアドレスにジャンプ(次の命令のアドレスを$raに)
    move    $ra, $s2        # mainを呼び出した元のアドレスを$raに復元
    j       $ra             # コンソールに戻る

test_prime:
    li      $t0, 2          # $t0に2を代入
    move    $t1, $a0        # $t1に$a0の値をコピー
prime_loop:
    rem     $t2, $t1, $t0   # $t1を$t0で割った余りを$t2に代入する
    beq     $t0, $t1, prime_r1 # $t0と$t1が等しいときprime_r1のアドレスにジャンプ
    beqz    $t2, prime_r2   # $t2が0ならprime_r2のアドレスにジャンプ
    addi    $t0, 1          # $t0に1を加える
    j       prime_loop      # prime_loopのアドレスにジャンプ
prime_r1:
    li      $v0, 1          # $v0に1を代入
    j       prime_end       # prime_endのアドレスにジャンプ
prime_r2:
    li      $v0, 0          # $v0に0を代入
prime_end:
    j       $ra             # 呼び出し元に戻る

print_int:
    li      $v0, 1          # $v0に1を代入
    syscall                 # システムコールの実行
    j       $ra             # 呼び出し元に戻る
print_str:
    li      $v0, 4          # $v0に4を代入
    syscall                 # システムコールの実行
    j       $ra             # 呼び出し元に戻る

    .data
    .align 2
space:
    .asciiz " "
    .align 2
line:
    .asciiz "\n"
\end{Verbatim}


\subsection{課題1-5で用いたコード} \label{sec:p1-5}
\begin{Verbatim}[numbers=left, xleftmargin=10mm, numbersep=6pt,
                    fontsize=\small, baselinestretch=0.8]
    .text
    .align 2
main:
    move    $s2, $ra        # $s2にmainを呼び出した元のアドレスを格納
    li      $s0, 2          # $s0に2を格納(検索する数)
    li      $s1, 0          # $s1に0を格納(素数の個数カウント)
    la      $s3, prime_array# $s3にprime_arrayのアドレスをロード
loop:
    move    $a0, $s0        # $s0の値を$a0にコピー
    jal     test_prime      # test_primeのアドレスにジャンプ(次の命令のアドレスを$raに)
    beqz    $v0, r1         # $v0が0ならば,r1に分岐
    # ここで配列にデータを格納
    move    $t0, $s1        # $s1の値を$t0にコピー
    mulo    $t0, $t0, 4     # $t0を4倍(4byte区切り)
    add     $t0, $t0, $s3   # $t0と$s3を足して$t0に格納
    sw      $s0, 0($t0)     # $s0の値を配列上にコピー
    addi    $s1, 1          # $s1に1を加える
r1:
    bgt     $s1, 100, loop2 # $s1が100より大きいとき,loop2にジャンプ
    addi    $s0, 1          # $s0に1を加える
    j       loop            # loopのアドレスにジャンプ
loop2:
    la      $a0, indicate   # indicateのアドレスを$a0にロード
    jal     print_str       # print_strのアドレスにジャンプ(次の命令のアドレスを$raに)
    #入力受付
    jal     read_int        # read_intのアドレスにジャンプ(次の命令のアドレスを$raに)
    # 例外確認
    bge     $v0, 101, end   # 入力が101以上の場合,終了
    ble     $v0, 0, end     # 入力が0以下の場合,終了
    # 正しい番地を計算
    move    $t0, $v0        # $v0の値を$t0にコピー
    addi    $t0, -1         # $t0から1を引く
    mulo    $t0, $t0, 4     # $t0を4倍(4byte区切り)
    add     $t0, $t0, $s3   # $t0と$s3を足して$t0に格納
    # 配列から読みだした値を表示
    lw      $a0, 0($t0)     # $s0の値を配列上にコピー
    jal     print_int       # print_intのアドレスにジャンプ(次の命令のアドレスを$raに)
    la      $a0, line       # lineのアドレスを$a0にロード
    jal     print_str       # print_strのアドレスにジャンプ(次の命令のアドレスを$raに)
    j       loop2
end:
    move    $ra, $s2        # mainを呼び出した元のアドレスを$raに復元
    j       $ra             # コンソールに戻る

test_prime:
    li      $t0, 2          # $t0に2を代入
    move    $t1, $a0        # $t1に$a0の値をコピー
prime_loop:
    rem     $t2, $t1, $t0   # $t1を$t0で割った余りを$t2に代入する
    beq     $t0, $t1, prime_r1 # $t0と$t1が等しいときprime_r1のアドレスにジャンプ
    beqz    $t2, prime_r2   # $t2が0ならprime_r2のアドレスにジャンプ
    addi    $t0, 1          # $t0に1を加える
    j       prime_loop      # prime_loopのアドレスにジャンプ
prime_r1:
    li      $v0, 1          # $v0に1を代入
    j       prime_end       # prime_endのアドレスにジャンプ
prime_r2:
    li      $v0, 0          # $v0に0を代入
prime_end:
    j       $ra             # 呼び出し元に戻る

print_int:
    li      $v0, 1          # $v0に1を代入
    syscall                 # システムコールの実行
    j       $ra             # 呼び出し元に戻る
print_str:
    li      $v0, 4          # $v0に4を代入
    syscall                 # システムコールの実行
    j       $ra             # 呼び出し元に戻る
read_int:
    li      $v0, 5          # read_int(戻り値は$v0に,1~100のみ入力可)
    syscall                 # 読み取り
    j       $ra             # 呼び出し元に戻る

    .data
    .align 2
indicate:
    .asciiz ">"
    .align 2
line:
    .asciiz "\n"
    .align 2
prime_array:
    .space 400
\end{Verbatim}

%--------------------------------------------------------------------%
% 参考文献
%   以下は,書き方の例である.実際に,参考にした書籍等を見て書くこと.
%   本文で引用する際は,\cite{book:algodata}などとすればよい.
\begin{thebibliography}{99}
  \bibitem{book:assembly} David A. Patterson,John L. Hennessy,コンピュータの構成と設計 第5版[下] ~ハードウエアとソフトウエア~,日経BP社,2014.
  %\bibitem{book:label2} 著者名,書名,出版社,発行年.
  %\bibitem{www:label3} WWWページタイトル,URL,アクセス日.
\end{thebibliography}

%--------------------------------------------------------------------%
%% 本文はここより上に書く(\begin{document}~\end{document}が本文である)
\end{document}
