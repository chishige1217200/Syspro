% ファイル先頭から\begin{document}までの内容(プレアンブル)については,
% 基本的に { } の中を書き換えるだけでよい.
\documentclass[autodetect-engine,dvi=dvipdfmx,ja=standard,
               a4j,11pt]{bxjsarticle}

%%======== プレアンブル ============================================%%
% 用紙設定:指示があれば,適切な余白に設定しなおす
\RequirePackage{geometry}
\geometry{reset,a4paper}
\geometry{hmargin=25truemm,top=25truemm,bottom=25truemm,footskip=10truemm}
%\geometry{showframe} % 本文の"枠"を確認したければ,コメントアウト

% 設定:図の挿入
% http://www.edu.cs.okayama-u.ac.jp/info/tool_guide/tex.html#graphicx
\usepackage{graphicx}

% 設定:ソースコードの挿入
% http://www.edu.cs.okayama-u.ac.jp/info/tool_guide/tex.html#fancyvrb
\usepackage{fancyvrb}
\renewcommand{\theFancyVerbLine}{\texttt{\footnotesize{\arabic{FancyVerbLine}:}}}

%%======== レポートタイトル等 ======================================%%
% ToDo: 提出要領に従って,適切なタイトル・サブタイトルを設定する
\title{システムプログラミング1 \\
       期末レポート}

% ToDo: 自分自身の氏名と学生番号に書き換える
\author{氏名: 重近 大智 (SHIGECHIKA, Daichi) \\
        学生番号: 09501527}

% ToDo: レポート課題等の指示に従って適切に書き換える
\date{出題日: 2020年10月05日 \\
      提出日: \today \\
      締切日: 2020年11月16日 \\}  % 注:最後の\\は不要に見えるが必要.


%%======== 本文 ====================================================%%
\begin{document}
\maketitle
% 目次つきの表紙ページにする場合はコメントを外す
%{\footnotesize \tableofcontents \newpage}

%% 本文は以下に書く.課題に応じて適切な章立てを構成すること.
%% 章=\section,節=\subsection,項=\subsubsection である.

%--------------------------------------------------------------------%
\section{概要} \label{sec:abstract}
本レポートでは,MIPS言語を用いて,提示された5つの課題に取り組み,その解答を報告する.実行結果はxspimによる結果である.

本レポートで報告するシステムプログラミング1の課題は次の5つである.
\begin{enumerate}
\item A.8節 「入力と出力」に示されている方法と, A.9節 最後「システムコール」に示されている方法のそれぞれで \verb|"Hello World"| を表示せよ.両者の方式を比較し考察せよ.\cite{book:assembly}
\item アセンブリ言語中で使用する \verb|.data|, \verb|.text| および \verb|.align| とは何か解説せよ. \ref{sec:p1-2}節のコード中の 9行目の \verb|.data| がない場合,どうなるかについて考察せよ.
\item A.6節 「手続き呼出し規約」に従って,再帰関数 \verb|fact| を実装せよ. (以降の課題においては,この規約に全て従うこと)\cite{book:assembly}
\item 素数を最初から100番目まで求めて表示するMIPSのアセンブリ言語プログラムを作成してテストせよ. その際,素数を求めるためにに表\ref{tab:routine}に示す2つのルーチンを作成すること.
\item 素数を最初から100番目まで求めて表示するMIPSのアセンブリ言語プログラムを作成してテストせよ. ただし,配列に実行結果を保存するように \verb|main| 部分を改造し, ユーザの入力によって任意の番目の配列要素を表示可能にせよ.
\end{enumerate}

\begin{table}[b]
\centering
	\caption{課題4で実装する2つのルーチン}
	\label{tab:routine}
    	\begin{tabular}{|l|l|}
	\hline
関数名&概要\\
	\hline
\verb|test_prime(n)|&\verb|n|が素数なら$1$,そうでなければ$0$を返す\\
	\hline
\verb|main()|&整数を順々に素数判定し,100個プリント\\
	\hline
	\end{tabular}
\end{table}

%--------------------------------------------------------------------%
\section{プログラムの説明}\label{sec:capp}
使用したMIPSアセンブリ言語のソースコードは,\ref{sec:makep}章に示す.

\subsection{課題1-1}
まず,A.8節 「入力と出力」に示されている方法で実装したコードについて説明する\cite{book:assembly}.このプログラムは起動すると,\verb|msg|ラベルに対応するメモリ上のアドレスを\verb|la|命令により,\verb|$a0|レジスタにロードする.次に\verb|move|命令を用いて,\verb|main|を呼び出したコンソールに戻るためのアドレスを\verb|$s0|レジスタに退避する.続いて,\verb|loop|ラベルから始まるループ部に突入する.ここでは\verb|lb|命令を用いて,\verb|$a0|レジスタが指すメモリ上のアドレスに記録されているバイトを\verb|$a1|レジスタにロードする.ロードしたときの値が$0$であった場合,\verb|beqz|命令での分岐条件を満たし,\verb|end|ラベルに対応するメモリ上のアドレスの命令にジャンプする.条件を満たさない場合は,\verb|jal|命令で\verb|print|ラベルにジャンプするとともに,\verb|$ra|レジスタに\verb|jal print|の次の命令が存在するメモリ上のアドレスを代入する.\verb|print|ラベルから始まる表示の処理では,まず\verb|lw|命令を用いて,メモリ上のアドレス\verb|0xffff0008|に存在するワードを\verb|$t0|レジスタにロードする.このとき,\verb|$t0|レジスタに入っている値は$1$かそれ以外であり,これが$1$であればプリンタは動作可能な状態である.\verb|li|命令で,即値$1$を\verb|$t1|レジスタにロードして,\verb|and|命令で\verb|$t0|レジスタと\verb|$t1|レジスタの論理積をとって\verb|$t0|レジスタに代入する.次の\verb|beqz|命令で\verb|$t0|レジスタの値が$0$である場合,プリンタの準備はできていないものとし,もう一度\verb|print|ラベルに対応するメモリ上のアドレスの命令に戻って準備を試みる.準備ができてい場合は,\verb|sw|命令でメモリ上のアドレス\verb|0xffff000c|に表示したい$1$文字の\verb|ASCII|コードを送る.送った瞬間にSPIMのコンソールに文字が表示される.そして,\verb|j $ra|により呼び出し元である\verb|main|中の\verb|jal print|の次の命令にジャンプする.ここまでで1文字表示できたので,次の文字情報を得るために\verb|addi|命令で\verb|$a0|レジスタの中身に1を加え,次に\verb|lb|命令でロードする\verb|ASCII|コードの参照アドレスを変化させる.そして\verb|j|命令で\verb|loop|のラベルにジャンプする.ループを抜けた\verb|end|ラベル部では,\verb|$s0|に退避していたコンソールに戻るためのアドレスを\verb|$ra|から復元し,\verb|j|命令でコンソールに戻る.

続いて,A.9節 最後「システムコール」に示されている方法で実装したコードについて説明する\cite{book:assembly}.このプログラムは起動すると,\verb|msg|ラベルに対応するメモリ上のアドレスを\verb|la|命令により,\verb|$a0|レジスタにロードする.次に\verb|syscall|命令を実行したときに参照されるシステム・コール・コードを\verb|$v0|レジスタに予めロードしておく.今回のシステム・コール・コードは$4$で,\verb|print_string|サービスを指定する.\verb|syscall|命令でシステムコールを実行し,文字列が出力される.文字列を出力できるため,アドレスを1文字ずつずらすような処理は行わない.\verb|j|命令でコンソールに戻る.

\subsection{課題1-2}
プログラムは,\verb|main|ラベルから動き始める.まず\verb|subu|命令でスタックを\verb|24|ビット確保する.そして,\verb|sw|命令でスタックに\verb|$ra|レジスタの値をバックアップする.\verb|jal|命令で\verb|_print_data|ラベルに対応するメモリ上のアドレスの命令にジャンプするとともに,\verb|$ra|レジスタに自身の次の命令のアドレスを格納する.\verb|_print_data|ラベルから始まる処理で,\verb|la|命令で\verb|ghost|ラベルに対応するメモリ上のアドレスを\verb|$a0|レジスタにロードする.\verb|lw|命令でワードを\verb|$a0|レジスタにロードする.そして\verb|li|命令で\verb|$v0|レジスタに即値$1$をロードする.\verb|syscall|命令でシステムコールを実行し,コンソールに表示する.\verb|j $ra|で\verb|main|ラベル内の処理に戻り,\verb|lw|命令でスタックから\verb|$ra|レジスタの値を復元する.そして,\verb|addu|命令で\verb|$sp|に$24$を加えてスタックを解放する.そして\verb|j|命令でコンソールに戻る.


\subsection{課題1-3} \label{sec:c1-3}
まず,\verb|move|命令を使って,\verb|$ra|レジスタの中身を\verb|$s0|レジスタに退避する.次に\verb|la|命令で\verb|msg|のラベルに対応するメモリ上のアドレスを\verb|a0|レジスタにロードする.\verb|jal|命令を使ってこの命令の次の命令に対応するメモリ上のアドレスを\verb|$ra|レジスタに保存し,ラベル\verb|print_str|に対応するメモリ上のアドレスの命令にジャンプする.\verb|print_str|のラベルからの処理では,\verb|li|命令で\verb|$v0|レジスタに$4$を代入して,\verb|syscall|命令でシステムコールを実行する.\verb|j $ra|で,\verb|print_str|を呼び出したその一つ次の命令に戻る.\verb|li|命令で\verb|$a0|レジスタに10を代入する.再び\verb|jal|命令で,この命令の次の命令に対応するメモリ上のアドレスを\verb|$ra|レジスタに保存し,ラベル\verb|fact|に対応するメモリ上のアドレスの命令にジャンプする.ここで\verb|fact|ラベルから始まる処理は,再起呼び出しを受ける可能性のある処理であるから自身を呼び出したときの\verb|$ra|レジスタの値や,再起呼び出しから戻る際に必要となる数値の情報はスタックに保存しておく必要がある.\verb|subu $sp, $sp, 32|から始まる5行の処理はスタックに関する命令である.まず,\verb|subu|命令でスタックポインタが指すメモリ上のアドレスを32バイト下げる.これによりスタックが32バイト確保された.続いて,\verb|sw|命令でメモリ上に対応するレジスタの値をコピーする.ここではスタックポインタを基準として20番地上の(つまりオフセットが20)アドレスに\verb|$ra|レジスタのコピー,16番地上のアドレスに\verb|$fp|レジスタのコピー,24番地上のアドレスに\verb|$a0|レジスタのコピーを取っている.なお,この\verb|$a0|レジスタはC言語におけるfact()関数の引数部に相当する値である.最後に\verb|addu|命令でフレームポインタを更新している.ここからは\verb|fact|ラベル部の内部処理に移る.まず,\verb|li|命令で\verb|$t0|レジスタに$1$を代入する.これは後の分岐命令で用いる値である.\verb|bgeu|命令で\verb|$a0|レジスタの値が\verb|$t0|レジスタより小さいとき,すなわち\verb|$a0|の値が$1$より小さいとき,\verb|else|のラベルに対応するメモリ上のアドレスの命令にジャンプする.そうでない場合は\verb|li|命令で\verb|$v0|レジスタに1を代入する.その後,\verb|lw|命令でスタックの内容を復元し,スタックを開放し,\verb|j $ra|で呼び出し元に戻る.\verb|else|ラベルにジャンプした場合は,\verb|addi|命令で\verb|$a0|レジスタに$-1$を加えて(すなわち$1$を引いて),\verb|jal|命令で\verb|fact|ラベルにジャンプする.再起呼び出しが終了し,戻ってくると40行目から下の\verb|lw|命令からスタックの復元と解放の処理が始まる.\verb|$ra|レジスタ,\verb|$fp|レジスタ,\verb|$a0|レジスタに対応する値を復元する.そして,\verb|fact|の計算のために,\verb|mulo|命令で\verb|$v0|レジスタと\verb|$a0|レジスタを乗算する.そして\verb|j $ra|で復元した\verb|$ra|レジスタの情報を元に呼び出し元に戻る.再起呼び出しから戻る過程で,\verb|$v0|レジスタは自身ともう一つのレジスタの値でしか更新されないようになっている.再起呼び出しから完全に戻り,ラベル\verb|main|まで戻ると,\verb|move|命令で\verb|$v0|レジスタの値を\verb|$a0|レジスタにコピーする.次に実行する\verb|syscall|命令の引数で\verb|$v0|レジスタの値は上書きされるため,別のレジスタにコピーするとともに実際に表示するときに使う\verb|$a0|レジスタにコピーしている.そして,\verb|jal|命令を使って,\verb|print_int|ラベルに対応するメモリ上のアドレスの命令にジャンプする.このとき,\verb|$ra|レジスタにこの次の命令のアドレスを格納する.\verb|print_int|のラベルからの処理では,\verb|li|命令で\verb|$v0|レジスタに$1$を代入して,\verb|syscall|命令でシステムコールを実行する.\verb|j $ra|で,\verb|print_int|を呼び出したその一つ次の命令に戻る.最後に改行するために\verb|la|命令で\verb|msg2|のアドレスをロードする.そして\verb|jal print_str|で\verb|print_str|ラベルに対応する命令にジャンプし,\verb|syscall|命令を実行する.最後に\verb|move|命令で\verb|$s0|レジスタからコンソールに戻るためのアドレスを\verb|$ra|から復元し,\verb|j|命令でコンソールに戻る.


\subsection{課題1-4} \label{sec:c1-4}
まず,\verb|move|命令を使って,\verb|$ra|レジスタの中身を\verb|$s2|レジスタに退避する.続いて,\verb|li|命令で\verb|$s0|レジスタに2,\verb|$s1|レジスタに1を代入する.このときの\verb|$s0|レジスタの値は検索する数,\verb|$s1|レジスタの値は素数の個数カウントに使用する.\verb|move|命令を使って,\verb|$s0|レジスタの値を\verb|$a0|レジスタにコピーする.そして,\verb|jal test_prime|で\verb|test_prime|ラベルに対応するメモリ上のアドレスの命令にジャンプするとともに,この次の命令のアドレスを\verb|$ra|レジスタに格納する.\verb|test_prime|ラベルから始まる処理では,まず\verb|li|命令で即値$2$を\verb|$t0|レジスタにロードする.そして,\verb|move|命令で\verb|$a0|レジスタの値を\verb|$t1|レジスタにコピーする.ここから先は\verb|prime_loop|ラベルの後の命令となり,ループされる処理である.まず,\verb|rem|命令で\verb|$t1|レジスタの値を\verb|$t0|レジスタの値で割って,その余りを\verb|$t2|レジスタに代入する.続いて分岐命令\verb|beq|で,\verb|$t0|レジスタと\verb|$t1|レジスタの値が等しいとき,すなわち素数であるとき,\verb|j|命令で\verb|prime_r1|ラベルに対応するメモリ上のアドレスの命令にジャンプする.その次の\verb|beqz|命令は,\verb|$t2|レジスタが$0$のとき,すなわち割り切れて素数ではないとき,\verb|prime_r2|ラベルに対応するメモリ上のアドレスの命令にジャンプする.いずれの条件にも合致しない場合は\verb|addi|命令で\verb|$t0|レジスタに$1$を加えて,\verb|j prime_loop|でループの初めにジャンプする.\verb|prime_r1|ラベルから始まる処理では,\verb|li|命令で\verb|$v0|レジスタに即値$1$をロードし,\verb|prime_end|ラベルにジャンプする.\verb|prime_r2|ラベルから始まる処理は,\verb|li|命令で\verb|$v0|レジスタに即値0をロードする.\verb|prime_end|の処理は,\verb|j $ra|により呼び出し元に戻る処理である.\verb|main|のループ部に処理が戻ってくると,\verb|beqz|命令が実行される.\verb|$v0|レジスタの値が$0$ならば,\verb|r1|ラベルに対応するメモリ上のアドレスの命令にジャンプする.\verb|r1|ラベル内の処理では,\verb|bge|命令で\verb|$s1|レジスタが100以上のときに,\verb|end|ラベルに対応するメモリ上のアドレスの命令にジャンプする.そうでなければ\verb|addi|命令で\verb|$s0|レジスタに1を加え,\verb|j loop|でループの初めにジャンプする.\verb|r1|に分岐しなかった場合は,数値が素数だったときで,\verb|addi|命令で\verb|$s1|レジスタに1を加える.そして,\verb|move|命令で\verb|$s0|レジスタの値を\verb|$a0|レジスタにコピーする.\verb|print_int|ラベルと\verb|print_str|ラベルに対応した処理が行われる.このとき,\verb|print_str|で表示する文字列は\verb|space|ラベルに対応するメモリ上のアドレスの文字列である.\verb|print_int|ラベルと\verb|print_str|ラベルの処理は,\ref{sec:c1-3}節で説明したものと同様の処理のため,ここでは説明しない.\verb|end|ラベルの処理では,\verb|la|命令で\verb|line|ラベルに対応するメモリ上のアドレスを\verb|$a0|レジスタにロードする.そして,\verb|print_str|で文字列を表示する.最後に\verb|move|命令で\verb|$s0|レジスタからコンソールに戻るためのアドレスを\verb|$ra|から復元し,\verb|j|命令でコンソールに戻る.

\subsection{課題1-5}
主に\ref{sec:c1-4}節とは異なる処理をしている部分に着目して説明する.

$7$行目で\verb|la|命令を使い,配列の先頭部分を指すメモリ上のアドレスを\verb|$s3|レジスタにロードする.12行目から始まる配列へのデータ格納処理は,まず\verb|move|命令で\verb|$s1|レジスタの値を\verb|$t0|レジスタにコピーする.次に\verb|mulo|命令でこの値を4倍する.ここまででオフセットの計算を行う.そして\verb|add|命令で\verb|$t0|レジスタと\verb|s3|レジスタの値を足して,配列の添字に対応するメモリ上のアドレスに変換する.この値を\verb|$t0|レジスタに代入する.そして\verb|sw|命令で\verb|$t0|の値に対応するメモリ上のアドレスに\verb|$s0|の値をコピーする.新たに追加された\verb|loop2|ラベルから始まる処理ではユーザからの入力を受け付ける.まず入力を受け付けていることを\verb|>|により示す.これを表示するために\verb|la|命令で\verb|indicate|ラベルに対応するメモリ上のアドレスを\verb|$a0|レジスタにロードする.次に\verb|print_str|に対応する処理を行う.そして,\verb|jal|命令で\verb|read_int|のアドレスにジャンプする.\verb|read_int|ラベルに対応する処理では,\verb|li|命令で$5$を代入し\verb|syscall|命令でシステムコールを実行する.このとき,入力された整数値は\verb|$v0|レジスタに代入される.\verb|j $ra|で呼び出し元である\verb|loop2|ラベル内の処理に戻る.\verb|bge|命令と\verb|ble|命令で例外用の分岐処理を行う.\verb|bge|命令は\verb|$v0|レジスタが101以上のとき,\verb|ble|命令は\verb|$v0|レジスタが0以下のとき,\verb|end|ラベルにジャンプする.そうでなければ\verb|$v0|レジスタの整数値を元に対応するメモリ上のアドレスを計算する.まず,\verb|$v0|レジスタの値を\verb|$t0|レジスタにコピーする.そして,\verb|$t0|レジスタから$1$を引く.次に\verb|mulo|命令でこの値を4倍する.そして\verb|add|命令で\verb|$t0|レジスタと\verb|s3|レジスタの値を足して,配列の添字に対応するメモリ上のアドレスに変換する.そして\verb|$t0|レジスタの情報を元にメモリ上の値にアクセスし,\verb|lw|命令で\verb|$a0|レジスタに値をロードする.そして\verb|print_int|に対応する処理を行う.続いて\verb|la|命令で\verb|line|に対応するメモリ上のアドレスを\verb|$a0|レジスタにロードし,\verb|print_str|に対応する処理を行う.そして\verb|j loop2|で\verb|loop2|の処理の先頭にジャンプする.素数であれば$1$を返す処理である45-60行の部分は,\ref{sec:p1-4}節に示す課題1-4のコードの26-41行目と全く同じ処理をするため,ここでは説明しない.\verb|end|ラベルに対応する処理では,最\verb|move|命令で\verb|$s2|レジスタからコンソールに戻るためのアドレスを\verb|$ra|から復元し,\verb|j|命令でコンソールに戻る.


%--------------------------------------------------------------------%
\section{プログラムの使用法と実行結果}\label{sec:howresult}

プログラムは,CentOS 7.6.1810 (Core) のxspimで動作を確認している.まず,ターミナルに\verb|xspim -mapped_io&|と打ち込んで,xspimを実行する.
実行後にloadの機能を使い,拡張子が\verb|.s|のアセンブリファイルを読み込む.runの機能で読み込んだプログラムを走らせる.プログラムを走らせた後,もう一度プログラムを走らせる場合には\verb|clear|でメモリとレジスタの値を初期化した後,再度ロードする必要がある.

\subsection{課題1-1}
A.8節 「入力と出力」に示されている方法と,A.9節 最後「システムコール」に示されている方法のいずれの方法でもSPIMのコンソールの出力は\verb|Hello World|となることを確認した.
\subsection{課題1-2}
\ref{sec:p1-2}節に示したコードを実行すると,コンソール出力は\verb|12|となる.それに対して,9行目の \verb|.data| がない場合は,コンソール出力は\verb|1212|となった.表示した後,プログラムは終了する.

\subsection{課題1-3}
実行すると,コンソール出力は\verb|The factorial of 10 is 3628800|となる.表示した後,プログラムは終了する.

\subsection{課題1-4}
実行すると,2から始まる合計100個の素数をコンソールに表示し,終了する.コンソールに表示される素数は次のとおりであり,適宜改行している.実際は$541$の表示後に一度だけ改行する.表示した後,プログラムは終了する.
\begin{verbatim}
2 3 5 7 11 13 17 19 23 29 31 37 41 43 47 53 59 61 67 71 73 79 83 89 97 101 103 107 109
113 127 131 137 139 149 151 157 163 167 173 179 181 191 193 197 199 211 223 227 229 233
239 241 251 257 263 269 271 277 281 283 293 307 311 313 317 331 337 347 349 353 359 367
373 379 383 389 397 401 409 419 421 431 433 439 443 449 457 461 463 467 479 487 491 499
503 509 521 523 541 
\end{verbatim}

\subsection{課題1-5}
実行すると,コンソールに\verb|>|が表示され,入力待ちになる.ここに$1$-$100$の数値を入力できる.入力後に\verb|Enter|キーを押すと,対応する素数が1つ表示される.$0$以下か$101$以上の値を入力するとプログラムが終了する.以下に入力と出力の一例を示す.
\begin{verbatim}
>23
83
>81
419
>0
\end{verbatim}

%--------------------------------------------------------------------%
\section{考察} \label{sec:review}

\subsection{課題1-1}
A.8節 「入力と出力」に示されている方法に示されている方法で実装する場合,空行を除くコードは25行である.
それに対して,A.9節 最後「システムコール」に示されている方法で実装する場合,空行を除くコードはたった12行となった\cite{book:assembly}.
これは前者が1文字ずつしか表示できないため,1文字ずつ読み込む処理とアドレスを$1$加える処理(12-16行目)がループになっていることと,プリンタを利用する際にプリンタが使用可能な状態になっていることを確認するために\verb|0xffff0008|番地から値を読み,$1$になっているか確認する処理(18-21行目)があることに起因する.また\verb|jal|命令を用いているため,\verb|main|を呼び出したときの\verb|$ra|レジスタの値が破壊されるため,そのコピーをするために\verb|move|命令を使っていることも行数が増える原因の一端となっている.後者は,文字列を表示できるため,アドレスに1を加えてずらす処理が必要なく,表示するにあたって特定番地の値の確認を取る必要もない.必要なのは,\verb|syscall|命令の動作を規定する引数を\verb|$v0|レジスタに事前に代入するだけである.そのため,非常に簡潔なコードとなっている.

特に 「入力と出力」に示されている方法で実装する場合には,\verb|0xffff0008|番地の確認を怠ると,プリンタの故障につながることがあったり,計算機によって異なるアドレスをとる可能性のある\verb|0xffff0008|番地をプログラムのコード内に書いてしまうことになったりするなどの問題もある.また,他のプログラムと同時に走行した場合の優先順序をつけられないために競合が発生することがある.システムコールを用いていれば,表示するタイミングや他のプログラムとの優先順序もオペレーティングシステムが決めてくれるため,プログラムを書く際に深く意識する必要はない.

\subsection{課題1-2}
\verb|.text|は,これに後続する項目をテキスト・セグメントの中に格納する指令である\cite{book:assembly}.テキスト・セグメントとは,プログラムの命令を格納する領域のことを指す.

\verb|.data|は,これに後続する項目をデータ・セグメントの中に格納する指令である\cite{book:assembly}.データ・セグメントは,テキスト・セグメント上に位置していて,特に静的データ・セグメントは開始アドレス\verb|0x10000000|から始まる領域に格納される.今回の\verb|.word 12|をデータ・セグメントに配置するようにした場合,このアドレスの領域に格納されることになる.

\verb|.align|は,これに後続するデータを$2^n$バイト境界に整列化する命令である\cite{book:assembly}.MIPS R2000の多くのロード・ストア命令は,整列化されたデータのみを操作の対象としているため,ワードのデータアドレスは4の倍数でなければならない.つまり\verb|.align 2|をすれば,$2^2$すなわち$4$バイト境界に整列化することになり,ロード・ストア命令が正しく実行できるようになる.また,SPIMシミュレータは\verb|MIPS32|アーキテクチャを実現したプロセッサ用に書かれたアセンブリ言語を実行するシミュレータであるため,ワード単位は\verb|32bit|すなわち$4$バイトである.このため,どの命令もデータも$4$バイト境界から始まっておくべきである.

\ref{sec:p1-2}節のコード中の9行目の\verb|.data|がない場合,\verb|.word 12|はテキスト・セグメント中に配置されることとなり,プログラムの一部となる.ジャンプ命令を実行しない場合は,プログラムカウンタを$4$バイトずつ増やしながら命令を実行していくため,\verb|.word 12|も命令であるかのように実行しようとする.\verb|MIPS32|アーキテクチャの命令は,$4$バイト長であり,\verb|.word 12|もバイト長は$4$バイトでそこに10進数の$12$があるだけである.同じ長さであるから命令として解釈することも可能になる.数値$12$を命令として解釈するとp.611の図A.10.2では\verb|funct(5:0)|の\verb|syscall|が対応する\cite{book:assembly}.これは12を下位2桁と考えることになるためである.これで,\verb|.word 12|で\verb|syscall|命令を実行し,その後の15行目でも\verb|syscall|命令を実行するため,2回\verb|syscall|命令を実行することになる.このため,コンソールの出力が\verb|1212|となった.もし\verb|.data|を書いていれば,開始アドレス\verb|0x10000000|から始まる静的データ・セグメント領域に格納されるため,プログラムの実行途中にプログラムカウンタに指されることは起こらない.そのため,コンソールの出力は\verb|12|となる.逆を言えば,プログラムカウンタに指されないことを確約できれば,テキスト・セグメント中に格納していても問題にならないとも考えられる.

%--------------------------------------------------------------------%
\section{感想}
アセンブリ言語を書くのは難しく感じたが,課題に取り組んでいるうちに使い方やプログラムの動きも分かるようになり,楽しかった.課題1-2を考察するのが大変だった.

%--------------------------------------------------------------------%
\section{作成したプログラムのソースコード} \label{sec:makep}

使用したプログラムを以下に添付する.
%なお,\ref{sec:abstract}章に示した課題については,
%\ref{xxxx}章で示したようにすべて正常に動作したことを付記しておく.

% Verbatim environment
% プリアンブルで \usepackage{fancyvrb} が必要.
%   - numbers           行番号を表示.left なら左に表示.
%   - xleftmargin       枠の左の余白.行番号表示用に余白を与えたい.
%   - numbersep         行番号と枠の間隙 (gap).デフォルトは 12 pt.
%   - fontsize          フォントサイズ指定
%   - baselinestretch   行間の大きさを比率で指定.デフォルトは 1.0.


\subsection{課題1-1で用いたコード} \label{sec:p1-1}
下記は,「入力と出力」に示されている方法で実装した例.
\begin{Verbatim}[numbers=left, xleftmargin=10mm, numbersep=6pt,
                    fontsize=\small, baselinestretch=0.8]
    .data                   # データセグメント
    .align 2                # 4バイト境界に揃える
msg:
    .asciiz "Hello World"   # 出力文字情報

    .text                   # テキストセグメント
    .align 2                # 4バイト境界に揃える
main:
    la      $a0, msg        # a0にmsgのアドレスを格納する
    move    $s0, $ra        # $s0にmainを呼び出した元のアドレスを格納
loop:
    lb      $a1, 0($a0)     # a0の指し先の値を$a1にロードする
    beqz    $a1, end        # a1が0のとき,endへ
    jal     print           # printのアドレスにジャンプ(次の命令のアドレスを$raに)
    addi    $a0, 1          # $a0に1を加える
    j       loop            # mainのアドレスにジャンプ
print:
    lw      $t0, 0xffff0008 # 0xffff0008番地にあるワードを$t0にロードする
    li      $t1, 1          # $t1に1を代入する
    and     $t0, $t0, $t1   # $t0と$t1の論理積をとって$t0に代入する($t0が1か確認する)
    beqz    $t0, print      # $t0が0のとき,もう一度準備を試みる
    sw      $a1, 0xffff000c # 0xffff000c番地に$a1にあるワードを送る
    j       $ra             # 呼び出し元に戻る
end:
    move    $ra, $s0        # mainを呼び出した元のアドレスを$raに復元
    j       $ra             # コンソールに戻る
\end{Verbatim}

下記は,「システムコール」に示されている方法で実装した例.
\begin{Verbatim}[numbers=left, xleftmargin=10mm, numbersep=6pt,
                    fontsize=\small, baselinestretch=0.8]
    .data                   # データセグメント
    .align 2                # 4バイト境界に揃える
msg:
    .asciiz "Hello World"   # 出力文字情報

    .text                   # テキストセグメント
    .align 2                # 4バイト境界に揃える
main:
    la      $a0, msg        # msg のアドレスを $a0 に格納
    li      $v0, 4          # print_stringを指定
    syscall                 # システムコールの実行
    j       $ra             # コンソールに戻る
\end{Verbatim}

\subsection{課題1-2で用いたコード} \label{sec:p1-2}
\begin{Verbatim}[numbers=left, xleftmargin=10mm, numbersep=6pt,
                    fontsize=\small, baselinestretch=0.8]
    .text                   # テキストセグメント
    .align 2                # 4バイト境界に揃える

_print_data:
    la      $a0, ghost      # ghostのアドレスを$a0にロード
    lw      $a0, 0($a0)     # $a0レジスタに対応するメモリ上のアドレスのワードを自身にロード
    li      $v0, 1          # $v0レジスタに1を代入

    .data                   # データセグメント
    .align 2                # 4バイト境界に揃える
ghost:
    .word 12                # 12(1ワード)

    .text                   # テキストセグメント
    syscall                 # システムコールの実行
    j       $ra             # 呼び出し元に戻る

main:                       # スタート
    subu    $sp, $sp, 24    # スタックを積む
    sw      $ra, 16($sp)    # $sp + 16番地のアドレスに$raの値をバックアップ
    jal     _print_data     # _print_dataのアドレスにジャンプ(次の命令のアドレスを$raに)
    lw      $ra, 16($sp)    # $sp + 16番地のアドレスから$raの値を復元
    addu    $sp, $sp, 24    # スタックを解放
    j       $ra             # 呼び出し元に戻る
\end{Verbatim}


\subsection{課題1-3で用いたコード} \label{sec:p1-3}
\begin{Verbatim}[numbers=left, xleftmargin=10mm, numbersep=6pt,
                    fontsize=\small, baselinestretch=0.8]
    .text                   # テキストセグメント
    .align 2                # 4バイト境界に揃える
main:
    move    $s0, $ra        # mainを呼んだ戻り先のアドレスを$s0に保存しておく

    la      $a0, msg        # msgのアドレスを$a0にロード
    jal     print_str       # print_strのアドレスにジャンプ(次の命令のアドレスを$raに)

    li      $a0, 10         # $a0に10を代入(任意のint値)

    jal     fact            # factに飛ぶ
    move    $a0, $v0        # $a0に$v0の値をコピー
    jal     print_int       # print_intのアドレスにジャンプ(次の命令のアドレスを$raに)

    la      $a0, msg2       # msg2のアドレスを$a0にロード
    jal     print_str       # print_strのアドレスにジャンプ(次の命令のアドレスを$raに)

    move    $ra, $s0        # mainを呼んだ戻り先のアドレスを$raに代入
    j       $ra             # コンソールに戻る

fact:
    subu    $sp, $sp, 32    # 32バイト確保
    sw      $ra, 20($sp)    # $sp + 20のアドレスにもともとの呼出しアドレスを保存
    sw      $fp, 16($sp)    # $sp + 16のアドレスに$fpを保存
    sw      $a0, 24($sp)    # $sp + 24のアドレスに$a0を保存
    addu    $fp, $sp, 28    # 新しく$fpを設定

    li      $t0,   1        # $t0に1を代入(分岐命令用)

    bgeu    $a0, $t0, else  # $a0が$t0より小さいとき,elseへ
    li      $v0, 1          # $v0に1を代入
    lw      $ra, 20($sp)    # 保存しておいた$spを復元
    lw      $fp, 16($sp)    # 保存しておいた$fpを復元
    addu    $sp, $sp, 32    # スタックを解放
    j       $ra             # 呼び出し元に戻る
else:
    addi    $a0, -1         # $a0から1を引く
    jal     fact            # 再帰呼び出しをする

    lw      $ra, 20($sp)    # 保存しておいた$spを復元
    lw      $fp, 16($sp)    # 保存しておいた$fpを復元
    lw      $a0, 24($sp)    # 対応する$a0の値を復元
    addu    $sp, $sp, 32    # スタックを解放

    mulo    $v0, $a0, $v0   # $v0と$a0をかけて$v0に代入
    j       $ra             # 呼び出し元に戻る

print_int:
    li      $v0, 1          # $v0に1を代入
    syscall                 # システムコールの実行
    j       $ra             # 呼び出し元に戻る
print_str:
    li      $v0, 4          # $v0に4を代入
    syscall                 # システムコールの実行
    j       $ra             # 呼び出し元に戻る

    .data                   # データセグメント
    .align 2                # 4バイト境界に揃える
msg:
    .asciiz "The factorial of 10 is "

    .align 2                # 4バイト境界に揃える
msg2:
    .asciiz "\n"
\end{Verbatim}

\subsection{課題1-4で用いたコード} \label{sec:p1-4}
\begin{Verbatim}[numbers=left, xleftmargin=10mm, numbersep=6pt,
                    fontsize=\small, baselinestretch=0.8]
    .text                   # テキストセグメント
    .align 2                # 4バイト境界に揃える
main:
    move    $s2, $ra        # $s2にmainを呼び出した元のアドレスを格納
    li      $s0, 2          # $s0に2を格納(検索する数)
    li      $s1, 0          # $s1に0を格納(素数の個数カウント)
loop:
    move    $a0, $s0        # $s0の値を$a0にコピー
    jal     test_prime      # test_primeのアドレスにジャンプ(次の命令のアドレスを$raに)
    beqz    $v0, r1         # $v0が0ならば,r1に分岐
    addi    $s1, 1          # $s1に1を加える
    move    $a0, $s0        # $s0の値を$a0にコピー
    jal     print_int       # print_intのアドレスにジャンプ(次の命令のアドレスを$raに)
    la      $a0, space      # spaceのアドレスを$a0にロード
    jal     print_str       # print_strのアドレスにジャンプ(次の命令のアドレスを$raに)
r1:
    bge     $s1, 100, end   # $s1が100以上のとき,endにジャンプ
    addi    $s0, 1          # $s0に1を加える
    j       loop            # loopのアドレスにジャンプ
end:
    la      $a0, line       # lineのアドレスを$a0にロード
    jal     print_str       # print_strのアドレスにジャンプ(次の命令のアドレスを$raに)
    move    $ra, $s2        # mainを呼び出した元のアドレスを$raに復元
    j       $ra             # コンソールに戻る

test_prime:
    li      $t0, 2          # $t0に2を代入
    move    $t1, $a0        # $t1に$a0の値をコピー
prime_loop:
    rem     $t2, $t1, $t0   # $t1を$t0で割った余りを$t2に代入する
    beq     $t0, $t1, prime_r1 # $t0と$t1が等しいときprime_r1のアドレスにジャンプ
    beqz    $t2, prime_r2   # $t2が0ならprime_r2のアドレスにジャンプ
    addi    $t0, 1          # $t0に1を加える
    j       prime_loop      # prime_loopのアドレスにジャンプ
prime_r1:
    li      $v0, 1          # $v0に1を代入
    j       prime_end       # prime_endのアドレスにジャンプ
prime_r2:
    li      $v0, 0          # $v0に0を代入
prime_end:
    j       $ra             # 呼び出し元に戻る

print_int:
    li      $v0, 1          # $v0に1を代入
    syscall                 # システムコールの実行
    j       $ra             # 呼び出し元に戻る
print_str:
    li      $v0, 4          # $v0に4を代入
    syscall                 # システムコールの実行
    j       $ra             # 呼び出し元に戻る

    .data                   # データセグメント
    .align 2                # 4バイト境界に揃える
space:
    .asciiz " "
    .align 2                # 4バイト境界に揃える
line:
    .asciiz "\n"
\end{Verbatim}


\subsection{課題1-5で用いたコード} \label{sec:p1-5}
\begin{Verbatim}[numbers=left, xleftmargin=10mm, numbersep=6pt,
                    fontsize=\small, baselinestretch=0.8]
    .text                   # テキストセグメント
    .align 2                # 4バイト境界に揃える
main:
    move    $s2, $ra        # $s2にmainを呼び出した元のアドレスを格納
    li      $s0, 2          # $s0に2を格納(検索する数)
    li      $s1, 0          # $s1に0を格納(素数の個数カウント)
    la      $s3, prime_array# $s3にprime_arrayのアドレスをロード
loop:
    move    $a0, $s0        # $s0の値を$a0にコピー
    jal     test_prime      # test_primeのアドレスにジャンプ(次の命令のアドレスを$raに)
    beqz    $v0, r1         # $v0が0ならば,r1に分岐
    # ここで配列にデータを格納
    move    $t0, $s1        # $s1の値を$t0にコピー
    mulo    $t0, $t0, 4     # $t0を4倍(4byte区切り)
    add     $t0, $t0, $s3   # $t0と$s3を足して$t0に格納
    sw      $s0, 0($t0)     # $s0の値を配列上にコピー
    addi    $s1, 1          # $s1に1を加える
r1:
    bgt     $s1, 100, loop2 # $s1が100より大きいとき,loop2にジャンプ
    addi    $s0, 1          # $s0に1を加える
    j       loop            # loopのアドレスにジャンプ
loop2:
    la      $a0, indicate   # indicateのアドレスを$a0にロード
    jal     print_str       # print_strのアドレスにジャンプ(次の命令のアドレスを$raに)
    #入力受付
    jal     read_int        # read_intのアドレスにジャンプ(次の命令のアドレスを$raに)
    # 例外確認
    bge     $v0, 101, end   # 入力が101以上の場合,終了
    ble     $v0, 0, end     # 入力が0以下の場合,終了
    # 正しい番地を計算
    move    $t0, $v0        # $v0の値を$t0にコピー
    addi    $t0, -1         # $t0から1を引く
    mulo    $t0, $t0, 4     # $t0を4倍(4byte区切り)
    add     $t0, $t0, $s3   # $t0と$s3を足して$t0に格納
    # 配列から読みだした値を表示
    lw      $a0, 0($t0)     # $a0に対応する配列上の値をロード
    jal     print_int       # print_intのアドレスにジャンプ(次の命令のアドレスを$raに)
    la      $a0, line       # lineのアドレスを$a0にロード
    jal     print_str       # print_strのアドレスにジャンプ(次の命令のアドレスを$raに)
    j       loop2
end:
    move    $ra, $s2        # mainを呼び出した元のアドレスを$raに復元
    j       $ra             # コンソールに戻る

test_prime:
    li      $t0, 2          # $t0に2を代入
    move    $t1, $a0        # $t1に$a0の値をコピー
prime_loop:
    rem     $t2, $t1, $t0   # $t1を$t0で割った余りを$t2に代入する
    beq     $t0, $t1, prime_r1 # $t0と$t1が等しいときprime_r1のアドレスにジャンプ
    beqz    $t2, prime_r2   # $t2が0ならprime_r2のアドレスにジャンプ
    addi    $t0, 1          # $t0に1を加える
    j       prime_loop      # prime_loopのアドレスにジャンプ
prime_r1:
    li      $v0, 1          # $v0に1を代入
    j       prime_end       # prime_endのアドレスにジャンプ
prime_r2:
    li      $v0, 0          # $v0に0を代入
prime_end:
    j       $ra             # 呼び出し元に戻る

print_int:
    li      $v0, 1          # $v0に1を代入
    syscall                 # システムコールの実行
    j       $ra             # 呼び出し元に戻る
print_str:
    li      $v0, 4          # $v0に4を代入
    syscall                 # システムコールの実行
    j       $ra             # 呼び出し元に戻る
read_int:
    li      $v0, 5          # read_int(戻り値は$v0に,1-100のみ入力可)
    syscall                 # 読み取り
    j       $ra             # 呼び出し元に戻る

    .data                   # データセグメント
    .align 2                # 4バイト境界に揃える
indicate:
    .asciiz ">"
    .align 2                # 4バイト境界に揃える
line:
    .asciiz "\n"
    .align 2                # 4バイト境界に揃える
prime_array:
    .space 400
\end{Verbatim}

%--------------------------------------------------------------------%
% 参考文献
%   以下は,書き方の例である.実際に,参考にした書籍等を見て書くこと.
%   本文で引用する際は,\cite{book:algodata}などとすればよい.
\begin{thebibliography}{99}
  \bibitem{book:assembly} David A. Patterson,John L. Hennessy,コンピュータの構成と設計 第5版[下] -ハードウエアとソフトウエア-,日経BP社,2014.
  %\bibitem{book:label2} 著者名,書名,出版社,発行年.
  %\bibitem{www:label3} WWWページタイトル,URL,アクセス日.
\end{thebibliography}

%--------------------------------------------------------------------%
%% 本文はここより上に書く(\begin{document}\UTF{FF5E}\end{document}が本文である)
\end{document}
